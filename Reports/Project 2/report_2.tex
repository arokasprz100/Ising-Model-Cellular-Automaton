\documentclass[12pt] {article}

%%% Preambuła %%%
\usepackage[T1]{fontenc}
\usepackage[polish]{babel}
\usepackage[utf8]{inputenc}
\usepackage{lmodern}
\usepackage[hidelinks]{hyperref}
\usepackage{mathptmx}
\usepackage{float}
\usepackage{graphicx}
\usepackage{amsmath}
\usepackage{xcolor}
\usepackage{listings}
\usepackage{geometry}
\usepackage{tocloft}
\usepackage{subcaption}
\usepackage{indentfirst}
\selectlanguage{polish}

\lstset{basicstyle=\ttfamily,
  showstringspaces=false,
  commentstyle=\color{red},
  keywordstyle=\color{blue},
  basicstyle=\ttfamily\footnotesize,
  columns=fullflexible,
  breaklines=true
}

\renewenvironment{abstract}
 {\small
  \begin{center}
  \bfseries \abstractname\vspace{-.5em}\vspace{0pt}
  \end{center}
  \list{}{%
    \setlength{\leftmargin}{5mm}% <---------- CHANGE HERE
    \setlength{\rightmargin}{\leftmargin}%
  }%
  \item\relax}
 {\endlist}

\renewcommand{\cftsecleader}{\cftdotfill{\cftdotsep}} 
\geometry{a4paper, total={170mm,257mm}, left=20mm, top=20mm,}

%%% Strona tytułowa %%%
\title {
	\large Automaty komórkowe \\
    \normalsize Projekt 2: model Isinga
    }

\author {Arkadiusz Kasprzak}
\date{}
	
\begin{document}


%%% Strona tytułowa %%%
\maketitle

%%% Streszczenie %%%
\begin{abstract}
Niniejszy dokument stanowi sprawozdanie z drugiego projektu realizowanego w ramach przedmiotu \textit{Automaty komórkowe}. Tematem projektu było opracowanie interaktywnej aplikacji internetowej implementującej ... 

\end{abstract}

%%% Spis treści %%%
\tableofcontents

\newpage 

\section{Wstęp}



\newpage

\begin{thebibliography}{9}

\bibitem{malarz}
  dr hab. inż. Krzysztof Malarz, prof. AGH,
  \emph{wykład prowadzony w ramach przedmiotu Automaty komórkowe},
  \url{http://home.agh.edu.pl/~malarz/dyd/ak/} (dostęp: 17.03.2020)

\end{thebibliography}

\end{document}